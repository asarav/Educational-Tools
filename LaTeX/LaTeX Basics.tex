\documentclass{article}

\author{Avinash Saravanan. This is the author.}
\title{My first {\LaTeX} Document. This is a title. This is of article document class. }
\date{11/22/18. This is the date.}

\begin{document}

\maketitle

\section{Introduction. This is how you make a section. }

This is some text!

This is a new line in the editor! You have to make two returns for a new paragraph. Text will automatically wrap.\hfill \break
This is a line break without indentation.

This is a line without hfill. \break

\section{Formatting}

This is normal text.
\textbf{This is bold text.}
\textit{This is italic.}
\emph{This is emphatic.}
\underline{This is underline.}
\underline{\textbf{This is underlined and bold}}


"This is in quotation marks."

``This is in proper quotation marks. Use ` at the beginning and ' at the end.''

\section {Numbering\label{numbering}}

\subsection{This is how you make a Subsection}
\subsection{Lists}
\textbf{This is an ordered list.}
\begin{enumerate}
\item Bread\label{bread}
\begin{enumerate}
\item sub entry 1
\item sub entry 2
\end {enumerate}
\item butter
\item cheese
\end{enumerate}

\textbf{This is an unordered list.}
\begin{itemize}
\item Corn
\begin {itemize}
\item sub entry 1
\item sub entry 2
\end {itemize}
\item Maize
\item Wheat
\end{itemize}

\section {References}

This is a reference to section \ref{list}.

This is a reference to section \ref{numbering}

This is a reference to an item in the ordered list \ref{bread}


\section {Labels\label{list}}

\end{document}